\documentclass[a4paper,twocolumn]{article}
\usepackage[utf8]{inputenc}
\usepackage{booktabs}
\usepackage{xcolor}
\usepackage{listings}
\usepackage{microtype}
\usepackage{syntax}
\usepackage{amsmath}
\usepackage[super]{nth}
\usepackage{tikz}
\usepackage{tikz-qtree}
\usetikzlibrary{shapes,patterns,positioning}
\usepackage[
  colorlinks,
  linkcolor={red!40!black},
  citecolor={blue!60!black},
  urlcolor={blue!60!black}
]{hyperref}

\title{Passgen}
\author{Patrick M. Elsen <pelsen@xfbs.net>}
\date{\today}

\begin{document}

\maketitle
\begin{abstract}
  Passwords are difficult. In this paper, a method by which passwords can be securely generated using a regular-expression-like language is presented.
\end{abstract}

\tableofcontents

\section{Introduction}



\section{Grammar}

The grammar that passgen patterns are defined in is a very limited and slightly modified subset of regular expressions. Notably, a lot of features are missing, and some (like \verb|\w|) have a changed meaning. The aim is that people familiar with regular expressions will be able to comprehend these patterns.

\setlength{\grammarindent}{4em}
\begin{grammar}

<pattern> ::= <group-inner>

<group> ::= ‘(’ <group-inner> ‘)’ \alt ‘(’ <group-inner> ‘)’ <repeat>

<range> ::= ‘[’ <range-inner> ‘]’ \alt ‘[’ <range-inner> ‘]’ <repeat>

<char> ::= ‘a’

<group-inner> ::= <segment> \alt <segment> <group-inner>

<segment> ::= <group> | <range> | <char>

<repeat> ::= ‘\{’ <number> ‘\}’ \alt ‘\{’ <number> ‘,’ <number> ‘\}’

\end{grammar}



\end{document}