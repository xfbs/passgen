% !TeX program = lualatex
% !Bib program = biber

\documentclass[a4paper,twocolumn]{article}
\usepackage[utf8]{inputenc}
\usepackage{booktabs}
\usepackage{xcolor}
\usepackage{listings}
\usepackage{microtype}
\usepackage{syntax}
\usepackage{amsmath}
\usepackage[super]{nth}
\usepackage[style=numeric,backend=biber]{biblatex}
\usepackage{tikz}
\usepackage{tikz-qtree}
\usetikzlibrary{shapes,patterns,positioning}
\usepackage[
  colorlinks,
  linkcolor={red!40!black},
  citecolor={blue!60!black},
  urlcolor={blue!60!black}
]{hyperref}

\addbibresource{passgen.bib}

\title{Passgen: Generating Passwords with a Regular-Expression-Like Language}
\author{Patrick M. Elsen <pelsen@xfbs.net>}
\date{\today}

\begin{document}

\maketitle
\begin{abstract}
	Passwords are commonly used for identification and authentication. Despite decades of research and recommendations, issues like password reuse or the use of weak, easily predicted passwords are rampant. In this paper, a method is presented by which passwords can be generated easily using a regular-expression-like language, allowing passwords fitting any conceivable password schemes can be generated. This language and the tool is presented and its utility is demonstrated. 
\end{abstract}

%\tableofcontents

\section{Introduction}

% what are passwords
% why do we need passwords
% what is wrong with passwords
% password manager use statistics




There are multiple ways to do authentication. One way is to use a biometric feature that is sensed (\emph{things you are}). This is done for example by fingerprint sensors on phones, or using cameras to detect faces. Another way is to use \emph{things you have}, such as a key or a \textsc{totp}\footnote{These are tokens that are usually generated by an app and that change every 30 or 60 seconds.} token. The last, and still most common method is to use \emph{things you know}, such as a password.

With \emph{things you are}, the limit is in how good sensors are at detecting fakes. It is possible to fool sensors into believing they are scanning a real human, when really it is a prosthetic finger with a cloned fingerprint, or fooling a face sensor with a 3D-printed face.

With \emph{things you have}, the danger is that these things can be lost. When a keyfile is lost, or a phone is stolen, it is then no longer possible to log into an account.

The last issue is that none of these method provide entropy that can be used, for example for encrypting things. It is not possible to generate a hash based on a fingerprint that is stable. 

Therefore, \emph{things you know}, meaning passwords, still are and will be a very prevalent method of authenticating to systems. However, passwords have some weaknesses, in that we seem to be bad at generating and remembering them.

When looking at public password dumps, it is obvious that password reuse and people using weak, easy to guess passwords is a very common problem. This is getting better, with people using password managers, but in the case of a password manager, the effective security is still limited by the master password.

Dell'Amico et al. showed in 2010 that\cite{5461951}. 


One way to remedy this is to generate passwords in a way that is secure, meaning that it is random and uses real entropy, but that are at the same time easy to remember for people. Because humans are very bad at remembering random strings of symbols, there are some methods by which passwords with a much higher entropy can be remembered than with random symbols.

In this project, a mechanism for generating passwords is presented. This is a small command-line utility which also provides a library (that can be used to embed this functionality into other programs) which allows generation of passwords using a \emph{patter}, similar to a regular expression. This means that the contents and shape of the password can be exactly specified.

The tool is able to use randomness from a variety of sources, and is able to calculate the effective entropy of generated passwords.

\section{Background}

% how passwords are used: hashed, scrypt, pbkdf
% how passwords are attacked

\subsection{Usage}

% how as passwords used

\subsection{Storage}

% as hash, meaning that there are potentially 256 bits of usable security, anything beyond is wasted

\subsection{Cracking}

% how are passwords attacked?

\subsection{Generation}

% how are passwords generated?
% review literature on generations

\subsection{Policies}

% what requirements are there for passwords?
% eg. banking, phone pin, etc.

\subsection{Managers}

% how are passwords stored? password manager. how many people use it?





\subsection{Regular Expressions}

\section{Generation}

In this paper

\section{Grammar}

The grammar that passgen patterns are defined in is a very limited and slightly modified subset of regular expressions. Notably, a lot of features are missing, and some (like \verb|\w|) have a changed meaning. The aim is that people familiar with regular expressions will be able to comprehend these patterns.

\setlength{\grammarindent}{4em}
\begin{grammar}

<pattern> ::= <group-inner>

<group> ::= ‘(’ <group-inner> ‘)’ \alt ‘(’ <group-inner> ‘)’ <repeat>

<range> ::= ‘[’ <range-inner> ‘]’ \alt ‘[’ <range-inner> ‘]’ <repeat>

<char> ::= ‘a’ | ‘\u{F6}’ | ‘\textbackslash [’

<group-inner> ::= <segment> \alt <segment> <group-inner>

<segment> ::= <segment-item> | <segment-item> <segment>

<segment-item> ::= <group> | <range> | <char> | <special>

<repeat> ::= ‘\{’ <number> ‘\}’ \alt ‘\{’ <number> ‘,’ <number> ‘\}’

<special> ::= ‘\textbackslash w[’ <identifier> ‘]’ <repeat>

<range-inner> ::= <range-item> | <range-item> <range-inner>

<range-item> ::= ‘a’ | ‘a-z’

\end{grammar}

\section{Usage}

\printbibliography

\end{document}